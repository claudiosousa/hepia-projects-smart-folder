\documentclass[11pt, a4paper]{article}

% Packages
\usepackage[french]{babel}
\usepackage[T1]{fontenc}
\usepackage[utf8]{inputenc}

\usepackage[left=2cm, right=2cm, top=2cm, bottom=2cm]{geometry}
\usepackage{fancyhdr}
\usepackage{lastpage}
\usepackage{hyperref}

\usepackage{float}

\usepackage{graphicx}
\graphicspath{{./img/}}
\usepackage{tikz}

% Reset paragraph indentation -------------------------------------------------
\setlength{\parindent}{0cm}

% Allow a paragraph to have a linebreak ---------------------------------------
\newcommand{\paragraphnl}[1]{\paragraph{#1}\mbox{}\\}

% Page header and footer ------------------------------------------------------
\pagestyle{fancy}
\setlength{\headheight}{33pt}
\renewcommand{\headrulewidth}{0.5pt}
\lhead{\includegraphics[height=1cm]{hepia.jpg}}
\chead{SmartFolder}
\rhead{Claudio Sousa - David Gonzalez}
\renewcommand{\footrulewidth}{0.5pt}
\lfoot{23 décembre 2016}
\cfoot{}
\rfoot{Page \thepage /\pageref{LastPage}}

% Table of contents depth -----------------------------------------------------
\setcounter{tocdepth}{3}

% Document --------------------------------------------------------------------
\begin{document}

\title
{
    \Huge{Programmation système} \\
    \Huge{SmartFolder}
}
\author
{
    \LARGE{David Gonzalez - Claudio Sousa}
}
\date{23 décembre 2016}
\maketitle

\begin{center}
    %\includegraphics[scale=0.27]{logo.png}
\end{center}

\thispagestyle{empty}

\newpage

% -----------------------------------------------------------------------------
\section{Introduction}

Ce TP de deuxième année en programmation système consiste à implémenter un programme similaire au SmartFolder sur MacOSX. \\

Le SmartFolder sur MacOSX recherche sur le disque des fichiers correspondant à un/des critères et
pour chacun des fichiers trouvés, le programme crée un lien dans un dossier specifié.

\subsection{Spécification fonctionnelle}
Ce programme possède deux modes de fonctionnement.

\subsubsection{Mode recherche}
Le premier est le mode \textit{recherche}.
C'est le mode par défaut qui simule le SmartFolder sur MacOSX.
Par ailleurs, le programme tourne indéfiniment tant qu'auncun signal d'arrêt n'est reçu. \\

Ce mode prend 3 paramètres:
\begin{itemize}
    \item \textit{<dir\_name>}: chemin où stocker les liens;
    \item \textit{<search\_path>}: chemin de recherche;
    \item \textit{<expression>}: critères de sélection. \\
\end{itemize}

\textit{<dir\_name>} et \textit{<search\_path>} sont de simples chemins vers des dossiers.\\
\textit{<expression>} correspond à une liste de critères dont l'interface est un sous-ensemble à celle de \textit{find}.

\paragraphnl{Expression}
Comme dit précédemment, l'interface est semblable à celle de \href{https://linux.die.net/man/1/find}{find}.
Voici la liste des options pris en charge:

\begin{itemize}
    \item \textit{...}: ...;
\end{itemize}

\paragraphnl{Output/comportement}

Si des liens symboliques existent, ils doivent être suivis. Des fichiers a doublent ne doivent pas apparaitre dupliqués dans le dossier de destination.

\subsubsection{Mode stop}
Le deuxième est le mode \textit{stop}.
Il permet d'arrêter une recherche en cours. \\

Ce mode prend 1 paramètre:
\begin{itemize}
    \item \textit{-d} \textit{<dir\_name>}: termine le SmatFolder pour le chemin specifié.
\end{itemize}

\newpage

% -----------------------------------------------------------------------------
\section{Development}
\subsection{Architecture}

\begin{figure}[H]
    \begin{center}
        \includegraphics[width=\textwidth]{modules.png}
    \end{center}
    \caption{Architecture du SmartFolder}
    \label{Architecture du SmartFolder}
\end{figure}

\subsubsection{Main}

Le programme principal a pour rôle de vérifier les arguments et de sélectionner le bon mode de fonctionnement.

Dans le mode \textit{recherche}, il a pour tâche de:
\begin{itemize}
    \item met le processus en arrière-plan;
    \item demande au module \textit{Parser} de traiter l'expression;
    \item initialise le module \textit{IPC};
    \item initialise et lance le module \textit{SmartFolder}.\\
\end{itemize}

Dans le mode \textit{stop}, son seul rôle est de signaler l'arrêt à l'autre instance (voir \nameref{sec:ipc}).

\subsubsection{SmartFolder}
\textit{SmartFolder} est le module principal qui va orchestrer la recherche et la mise à jour du dossier de destination. \\

Lorsque lancé, il va continuellement utiliser le module \textit{Finder} pour rechercher les fichiers correspondant au critère,
puis donner la liste des fichiers retournée au module \textit{Linker} pour qu'il mette à jour le répertoire de destination.

A noter qu'entre chaque recherche, il y a une pause de quelques secondes. \\

A l'arrêt, il est chargé de détruire le répertoire de destination et de libérer ses ressources.

\subsubsection{Parser}
\textit{Parser} est le module qui transforme
l'expression spécifiée dans la ligne de commande (critères de recherche) en une structure interne utilisable par le module \textit{Validator}.

\subsubsection{Validator}
Ce module vérifie si un fichier est valide selon l'expression créée par le module \textit{Parser}.

\subsubsection{Finder}
Ce module est responsable de produire la liste de tous les fichiers du dossier de recherche respectant les criteres de recherche.

La verification de l'expression est deleguée au module \textit{Validator}.

\subsubsection{Linker}
Le module \textit{Linker} a pour rôle de mettre à jour le dossier de destination
à l'aide d'une liste de fichiers passée en paramètre. \\

Pour chaque fichier, il crée un lien si celui-ci n'existe pas. Il efface les liens qui ne sont plus valides.

\subsubsection{IPC}
\label{sec:ipc}
Ce module a pour but de répondre au problème du mode \textit{stop}.
En effet, lorsqu'une instance de \textit{SmartFolder} souhaite arrêter une autre instance en cours d'exécution,
il faut d'une manière ou d'une autre permettre une communication simple entre les deux processus afin
qu'une instance puisse signaler un arrêt à une autre instance. \\

Le moyen de communication choisi a été les signaux POSIX, le signal \textit{SIGTERM} pour être plus précis.
Une instance qui veut donc signaler l'arrêt doit envoyer le signal cité au processus concerné. \\

Afin de pouvoir lancer un signal, il faut que le PID du processus cible soit connu.
Pour cela, le PID d'une instance en mode \textit{recherche} est stocké dans un fichier
dans le répertoire utilisateur.

Comme une instance de \textit{SmartFolder} est unique par dossier de destination,
ce fichier se nommera d'après le chemin de ce répertoire.

\subsubsection{IO}
Le but de ce module est d'offrir une interface simple aux appels systèmes et de factoriser la gestion des erreurs.

\subsubsection{Logger}
MOdule optionel qui factorise la gestion de logs si la fonctionalité est implémentée.

\end{document}
