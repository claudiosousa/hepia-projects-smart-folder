\documentclass[11pt, a4paper]{article}

% Packages
\usepackage[french]{babel}
\usepackage[T1]{fontenc}
\usepackage[utf8]{inputenc}

\usepackage[left=2cm, right=2cm, top=2cm, bottom=2cm]{geometry}
\usepackage{fancyhdr}
\usepackage{lastpage}
\usepackage{hyperref}

\usepackage{float}

\usepackage{graphicx}
\graphicspath{{./img/}}
\usepackage{tikz}

% Reset paragraph indentation -------------------------------------------------
\setlength{\parindent}{0cm}

% Allow a paragraph to have a linebreak ---------------------------------------
\newcommand{\paragraphnl}[1]{\paragraph{#1}\mbox{}\\}

% Page header and footer ------------------------------------------------------
\pagestyle{fancy}
\setlength{\headheight}{33pt}
\renewcommand{\headrulewidth}{0.5pt}
\lhead{\includegraphics[height=1cm]{hepia.jpg}}
\chead{SmartFolder}
\rhead{Claudio Sousa - David Gonzalez}
\renewcommand{\footrulewidth}{0.5pt}
\lfoot{23 décembre 2016}
\cfoot{}
\rfoot{Page \thepage /\pageref{LastPage}}

% Table of contents depth -----------------------------------------------------
\setcounter{tocdepth}{3}

% Document --------------------------------------------------------------------
\begin{document}

\title
{
    \Huge{Programmation concurrente} \\
    \Huge{SmartFolder}
}
\author
{
    \LARGE{David Gonzalez - Claudio Sousa}
}
\date{23 décembre 2016}
\maketitle

\begin{center}
    %\includegraphics[scale=0.27]{logo.png}
\end{center}

\thispagestyle{empty}

\newpage

% -----------------------------------------------------------------------------
\section{Introduction}

Ce TP de deuxième année en programmation système consiste à implémenter un programme similaire au SmartFolder sur MacOSX. \\

Le SmartFolder sur MacOSX recherche sur le disque des fichiers correspondant à un/des critères et
pour chacun des fichiers trouvés, le programme crée un lien dans un dossier specifié.

\subsection{Spécification fonctionnelle}
Ce programme possède deux modes de fonctionnement.

\subsubsection{Mode recherche}
Le premier est le mode \textit{recherche}.
C'est le mode par défaut qui simule le SmartFolder sur MacOSX.
Par ailleurs, le programme tourne indéfiniment tant qu'auncun signal d'arrêt n'est reçu. \\

Ce mode prend 3 paramètres:
\begin{itemize}
    \item \textit{<dir\_name>}: chemin où stocker les liens;
    \item \textit{<search\_path>}: chemin de recherche;
    \item \textit{[expression]}: critères de sélection. \\
\end{itemize}

\textit{<dir\_name>} et \textit{<search\_path>} sont de simples chemin vers des dossiers.
Concernant \textit{[expression]}, ceci correspond à une liste de critères dont l'interface est identique à celle de \textit{find}.

\paragraphnl{Expression}
...

\subsubsection{Mode stop}
Le deuxième est le mode \textit{stop}.
Il permet d'arrêter une recherche en cours. \\

Ce mode prend 2 paramètres:
\begin{itemize}
    \item \textit{-d}: paramètre qui permet de lancer ce mode;
    \item \textit{<dir\_name>}: chemin traité par le SmartFolder à terminé.
\end{itemize}

\newpage

% -----------------------------------------------------------------------------
\section{Development}
\subsection{Architecture}

\begin{figure}[H]
    \begin{center}
        \includegraphics[width=\textwidth]{modules.png}
    \end{center}
    \caption{Architecture du SmartFolder}
    \label{Architecture du SmartFolder}
\end{figure}

\subsubsection{Main}

Le programme principal a pour rôle de vérifier les arguments et de sélectionner le bon mon de fonctionnement.

Dans le mode \textit{recherche}, il a pour tâche de:
\begin{itemize}
    \item mettre le processus en arrière-plan;
    \item traiter l'expression;
    \item poser le \textit{pidfile};
    \item initialiser et lancer la recherche.
\end{itemize}

Lorsqu'il reçoit un signal d'arrêt, il libère la mémoire, efface le \textit{pidfile} et ce termine. \\

Dans le mode \textit{stop}, son seul rôle est de signaler l'arrêt (voir \nameref{sec:ipc}).

\subsubsection{SmartFolder}

\subsubsection{Parser}

\subsubsection{Validator}

\subsubsection{Finder}

\subsubsection{Linker}

\subsubsection{IPC}
\label{sec:ipc}

\subsubsection{IO}

\subsubsection{Logger}

\end{document}
